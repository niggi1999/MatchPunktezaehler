\chapter{Lösungsstrategie}

\section{Einstieg}
Die folgende Tabelle stellt die Qualitätsziele für den Matchpunktezähler mit den passenden Architekturansätzen gegenüber, und erleichter den Einstieg in die Lösung.\\[0.0cm]
\begin{center}
\begin{tabular}[h]{|l|l|}
\cline{1-2}
\textbf{Qualitätsziele} & \textbf{Dem zuträgliche Architekturansätze}\\
\cline{1-2}
Möglichkeit für Nachbau &
\tabitem Verwenden eines Raspberry Pi\\
& \tabitem Nutzen günstiger Nutzerhardware\\
& \tabitem Anzeige über Webserver\\
\cline{1-2}
Anpassungsmöglichkeiten &
\tabitem Aufgliederung in Front und Backend\\
& \tabitem Eingene Klasse für Spielregeln\\
& \tabitem Eigene Klasse für die Bluetooth-Schnittstelle\\
\cline{1-2}
Einheitliche Programmiersprache &
\tabitem Pythen bietet zahlreiche Plugins\\
& \tabitem Webserver mit Flask möglich\\
& \tabitem Bluetooth-Signale interpretierbar\\
\cline{1-2}
Gute Portabilität &
\tabitem Akkubetrieb\\
& \tabitem verwenden von Bluetooth LE\\
\cline{1-2} 
\end{tabular}
\end{center}
\section{Aufbau}
Der Matchpunktezähler ist als Python Programm in Kombination mit Flask als Webframework realisiert. Es besteht aus den folgenden Teilen:

\begin{itemize}
	\item Implementierung der Spielregeln
	\item Interpretation der Bluetooth-Signale
	\item Eine Benutzeroberfläche die von einem Webserver abgerufen wird
	\item Verbindung des Backends und des Frontends
\end{itemize}

Die Aufteilung ermöglicht es verschiedene Spiele für den den Matchpunktezähler zu implementieren und die durch eine Auswahl anwählbar zu machen. Außerdem kann die Interpretation der Bluetooth-Signale bei Verwendung von anderer Hardware einfacher angepasst oder ausgetauscht werden.\\
Bei einer notwendigen Anpassung der Benutzeroberfläche kann dieses Aufgrund der Schnittstelle einfach ausgetauscht werden.

\section{Spielstrategie}

\section{Anbindung}