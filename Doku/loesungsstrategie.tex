\chapter{Lösungsstrategie}

\section{Einstieg}
Die folgende Tabelle stellt die Qualitätsziele für den Matchpunktezähler mit den passenden Architekturansätzen gegenüber, und erleichtert den Einstieg in die Lösung.\\[0.0cm]
\begin{center}
\begin{tabular}[h]{|l|l|}
\cline{1-2}
\textbf{Qualitätsziele} & \textbf{Dem zuträgliche Architekturansätze}\\
\cline{1-2}
Möglichkeit für Nachbau &
\tabitem Verwenden eines Raspberry Pi\\
& \tabitem Nutzen günstiger Nutzerhardware\\
& \tabitem Anzeige über Webserver\\
\cline{1-2}
Anpassungsmöglichkeiten &
\tabitem Aufgliederung in Front und Backend\\
& \tabitem Eingene Klasse für Spielregeln\\
& \tabitem Eigene Klasse für die Bluetooth-Schnittstelle\\
\cline{1-2}
Einheitliche Programmiersprache &
\tabitem Pythen bietet zahlreiche Plugins\\
& \tabitem Webserver mit Flask möglich\\
& \tabitem Bluetooth-Signale interpretierbar\\
\cline{1-2}
Gute Portabilität &
\tabitem Akkubetrieb\\
& \tabitem verwenden von Bluetooth LE\\
\cline{1-2} 
\end{tabular}
\end{center}
\section{Aufbau}
Der Matchpunktezähler ist als Python Programm in Kombination mit Flask realisiert. Die Benutzeroberfläche des Frontends ist mit React realisiert. 
Es besteht aus den folgenden Teilen:

\begin{itemize}
	\item Implementierung der Spielregeln
	\item Interpretation der Bluetooth-Signale
	\item Eine Benutzeroberfläche die von einem Webserver abgerufen wird
	\item Verbindung der App und des Frontends
\end{itemize}

Die Aufteilung ermöglicht es verschiedene Spiele für den den Matchpunktezähler zu implementieren und diese anwählbar zu machen. Außerdem kann die Interpretation der Bluetooth-Signale bei Verwendung von anderer Hardware einfacher angepasst oder ausgetauscht werden.\\
Bei einer Anpassung der Benutzeroberfläche muss die Schnittstelle nur angepasst werden, wenn die Benutzeroberfläche durch die Anpassung weitere Daten benötigt.\\[0.4cm]

Zentrales Element beim Entwurf der Datenstruktur ist die Spielsituation eines Badminton Spiels: Welcher Spieler hat Aufschlag, welches Team steht auf welcher Seite des Spielfeldes, welcher Spieler steht bei einem Doppel wo und wie ist der aktuelle Spielstand.
\section{Spielstrategie}
Um den Ablauf eines Badminton Spiels zu bestimmen, muss zunächst ein Anfangszustand bestimmt werden. Im Anschluss kann mit jedem erzielten Punkt die weitere Vorgehensweise bestimmt werden. Das Programm reagiert nur auf die gegebenen Eingaben, kann aber bei einer Fehleingabe mit dem UNDO-Mechanismus in den vorigen Zustand zurückversetzt werden.
\section{Anbindung}
Der Machtpunktezähler wird über eine Bluetooth-Schnittstelle mit einem Bluetooth-Eingabegerät verbunden, über welches die Eingabe an das Programm getätigt werden. Die Ausgabe an den Benutzer erfolgt über eine Website, die von mehreren Geräten abgerufen werden kann. Bei jeder Änderung wird die Anzeige automatisch aktualisiert.