\chapter{Qualitätsszenarien}

\section{Bewertungsszenarien}
\begin{center}
\begin{tabular}[h]{|l|l|}
\cline{1-2}
\textbf{Nr.} & \textbf{Szenario}\\
\cline{1-2}
1 & Ein Interessierter mit Grundkenntnissen in UML und Badminton möchte einen\\& Einstieg in die Architektur vom Match-Punktezähler finden. Lösungsstrategie\\&und Entwurf erschließen sich ihm innerhalb von 20 Minuten.\\ 
\cline{1-2}
2 & Ein Entwickler implementiert die Logik für ein weiteres Hallenspiel. Er findet\\&dafür eine Schnittstelle im Vorhanden Code, aus welcher sich die benötigte\\&Funktionalität erschließt.\\ 
\cline{1-2}
3 & Ein Badmintonspieler bringt seinen Match-Punktezähler mit ans Spielfeld.\\&Er ist innerhalb von maximal drei Minuten spielbereit.\\ 
\cline{1-2}
4 & In einem Badmintonspiel erzielt eine Mannschaft einen Punkt. Auf dem Display\\&wird die aufschlagende Position angegeben.\\ 
\cline{1-2}
5 & In einem Badmintondoppel erzielt die aufschlagende Mannschaft mehrere\\&Punkte hintereinander. Dies Spieler verlieren die Übersicht über ihre\\&Standposition beim Aufschlag, können diese aber schnell und einfach auf dem\\&Display ablesen. \\ 
\cline{1-2}
6 & Beim Zählvorgang ist eine falsche Eingabe getätigt worden. Eingaben können\\&jederzeit über einen „Zurück“ Knopf rückgängig gemacht werden. Die\\&Spielerpositionen werden auch hier korrekt angezeigt.\\ 
\cline{1-2}
7 & Ein Bediener betätigt eine Taste auf der Fernbedienung. Innerhalb von zwei\\&Sekunden wird seine Eingabe auf dem Display angezeigt.\\ 
\cline{1-2}
8 & Ein Benutzer verwendet ein altes IPad 2 mit veralteter Software. Der Zähler\\&funktioniert auch mit dem alten Browser.\\ 
\cline{1-2}
9 & Ein Badmintonspieler mit eigenem Raspberry Pi möchte sich selbst einen Match-\\&Punktezähler bauen. Er kann sich mithilfe einer Anleitung innerhalb eines\\&Nachmittags einen Match-Punktezäher zusammenbauen. \\ 
\cline{1-2}
\end{tabular}
\end{center}
\section{Qualitätsbaum}
Das folgende Bild gibt einen Überblick über die relevanten Qualitätsmerkmale und den ihnen jeweils zugeordneten Szenarien. 

\begin{figure}[h]
\begin{center}
\includegraphics[scale=0.7]{Grafiken/Qualität.pdf}
\caption{Bausteinansicht Ebene 1}
\end{center}
\end{figure}