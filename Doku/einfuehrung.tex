\chapter{Einführung und Ziele}
In vielen Ballsportarten müssen im Wettbewerb die Punkte eines Teams gezählt werden. Im Profisport wird dazu meist mindestens ein Schiedsrichter eingesetzt. Im Amateursport allerdings fehlt dieser. Aus diesem Grund müssen die Punkte meist von den Spielern selbst gezählt werden. Um diese Aufgabe zu erleichtern ist es wünschenswert, eine technische Hilfe dafür einzusetzen. 
\section{Aufgabenstellung}
Das Projekt Match-Punktezähler besitzt die Aufgabe eine kostengünstige Lösung zum Zählen und Darstellen von Punkten in Ballsportarten zu bieten. Das Projekt soll sportbegeisterten eine einfache und günstige Möglichkeit geben, einen eigenen Match-Punktezähler zu bauen oder zu erweitern. Der Punktezähler hilft dabei sich voll aufs Spielen zu konzentrieren.  
\section{Qualitätsziele}
Im Folgenden sind die wichtigsten Qualitätsziele aufgeführt:\\
\begin{itemize}
	\item Die Bedienung des Systems soll möglichst einfach und schnell funktionieren, sodass möglichst einfach und schnell mit dem Spielen begonnen werden kann. Es soll keinen langen und komplizierten Einrichtungsprozess benötigt werden. 
	\item Die Bedienung soll über eine Fernbedienung stattfinden. Der Zähler soll von einem Spieler innerhalb des Spielfeldes bedient werden können. 
	\item Das Programm soll auch auf älterer und eventuell nicht weiter unterstützter Hardware laufen, sodass alte Geräte wie z.B. alte Tablets, wie ein IPad 2, noch eine Verwendung finden können.
	\item Das Programm soll leicht um weitere Sportarten erweitert werden können. Die primäre Sportart ist Badminton. 
\end{itemize}
\section{Stakeholder}
\begin{center}
\begin{tabular}[h]{|l|l|}
\cline{1-2}
\textbf{Rolle} & \textbf{Erwartungshaltung}\\
\cline{1-2}
Auftraggeber &\tabitem Wünscht sich ein funktionierendes Produkt\\ 
&\tabitem Möchte eine Adäquate Dokumentation des Projektes.\\
&\tabitem Spielt selber Badminton\\
\cline{1-2}
Entwickler &\tabitem Nehmen Architekturaufgaben wahr\\ 
&\tabitem Führen ein Softwareprojekt für ihr Studium durch\\ 
&\tabitem Lernen durch das Projekt die praktische Softwareentwicklung.\\
&\tabitem Haben nur geringe Badmintonkenntnisse\\
\cline{1-2}
Andere &\tabitem Möchten die Funktion des Match-Punktezählers verstehen.\\
Leser &\tabitem Möchten sich selbst einen Match-Punktezähler bauen.\\
&\tabitem Möchten den Match-Punktezähler für ihre Sportart implementieren\\
& ~~~~und verwenden.\\
\cline{1-2}
\end{tabular}
\end{center}