\chapter{Entwurfsentscheidungen}
\section{Frontend}
Das Frontend wurde mit dem Javascript Framework React erstellt. Aufgrund der zeitlichen Begrenzung bei der Programmierung der Software, wurde React als Javascript Framework gewählt, da React eine niedrige Einstiegshürde aufweist. 

Die Struktur des Frontends entspricht einer Ablaufkette, welche durch die vom Backend übergebenen Daten gesteuert wird. Diese Ablaufkette ist in der Funktion Procedure implementiert, welche im Listing \ref{procedure} dargestellt wird.

\begin{lstlisting}[caption={Procedure Funktion des Frontends},captionpos=b]
import React from "react";
import Game from './game';
import GameMenu from "./gameMenu";
import Init from './init';
import PlayerMenu from "./playerMenu";
import NameMenu from "./nameMenu";
import LeaveGame from "./leaveGame";
import ChangeSide from './changeSide';

function Procedure (props){
    switch(props.data.status) {
        case 'init':
            return <Init data={props.data}/>;
        case 'playerMenu':
            return <PlayerMenu data={props.data}/>;
        case 'nameMenu':
            return <NameMenu data={props.data}/>;
        case 'gameMenu':
            return <GameMenu data={props.data}/>;
        case 'game':
            return <Game data={props.data}/>;
        case 'leaveGame':
            return <LeaveGame data={props.data}/>;
        case 'changeSide':
            return <ChangeSide data={props.data}/>;
        default:
            return <h1>Invalid status. Check spelling</h1>;
    }
}

export default Procedure
\end{lstlisting}
\label{procedure}

Procedure bekommt beim Aufruf die Daten des Backends übergeben, welche in Abhängigkeit des Status variieren. Der Status ist ein String der vom Backend an das Frontend übergeben wird. Dieser String definiert auf welcher Seite der Benutzeroberfläche sich der Benutzer befindet. In Listing \ref{procedure} in Zeile 11 ist dargestellt, dass die Wahl der Seite durch ein Switch Case der Status Variable erfolgt. Entspricht die Status Variable z.B. dem String \glqq init\grqq{} , so wird die Init-Seite der Benutzeroberfläche dargestellt. Diese Implementierung wird verwendet, da so die Steuerung der Software in der Verantwortung des Backends liegt. Außerdem können auf diese Weise sehr leicht neue Seiten der Benutzeroberfläche hinzugefügt werden, da lediglich das Switch Case erweitert werden muss.

\section{Backend}
Für das Backend wurde die Programmiersprache Python verwendet, da es sich hierbei um eine populäre Serversprache
handelt. Als Framework kam dabei \glqq Flask\grqq{} zum Einsatz, da es relativ lightweight ist und alle nötigen
Funktionalitäten bietet. Ein weiterer Grund für Flask war die dafür zur Verfügung stehende Bibliothek \glqq
flask-sse\grqq{}, die die Arbeit mit Server-Sent-Events vereinfacht. \newline Die Server-Sent-Events sind essenziell
für die Kommunikation mit dem Frontend, da die Seite nicht andauernd neu geladen werden soll. Die \glqq
flask-sse\grqq{} Bibliothek funktioniert nur im Zusammenhang mit einem Flask Request Context. Es muss also zum
auslösen eines SSE Events eine HTTP Anfrage vorliegen. Um diese Anfragen zu erstellen wird der HTTP-Client \glqq
httpx\grqq{} verwendet. Dieser ist sehr populär und bietet die Möglichkeit asynchron Anfragen zu stellen. \newline
Für die Bluetooth Kommunikation wird die Bibliothek \glqq evdev\grqq{} verwendet, da eine relativ einfache
Möglichkeit bietet um asynchron Bluetooth Signale zu empfangen. Außerdem lässt sich einfach ein Gerät verbinden.
\newline Für Tests wird die Bibliothek \glqq unittest\grqq{} verwendet, da sie umfangreiche Features bietet und mit
in der Standardbibliothek enthalten ist. \newline Alle Model Klassen erben von einer Abstrakten Oberklasse \glqq
abstractModel\grqq{}. Diese bietet eine Schnittstelle um empfangene Tastendrücke weiterzugeben. Wenn ein Objekt über
die \glqq attach\grqq{} Methode subscribed hat, so wird es bei Änderungen im Model informiert. Anschließend kann mit
\glqq getPublishMethod\grqq{} eine Methode erhalten werden, die die geänderten Daten auf den SSE-Stream published.
\newline Die Bluetooth Kommunikation läuft in einem separaten Thread ab, da parallel auf Tastendrücke und neue
HTTP Anfragen reagiert werden muss.