\chapter{Randbedingungen}
\section{Technische Randbedingungen}
Als Hardwaregrundlage wird in diesem Fall ein Raspberry Pi, sowie ein altes und ausgemustertes Tablet verwendet werden. Der Raspberry Pi fungiert hierbei als Webserver und Sendet die Informationen über ein Ad-hog Netzwerk an das Tablet oder ein beliebiges anders Endgerät. Um Eingaben in das System zu erleichtern wird eine Bluetooth Low Energy Fernbedienung verwendet. Diese besitzt 5 Tasten um Eingaben im Programm zu tätigen. Außerdem kann sie auf verschiedenste Weise am Spieler angebracht werden.
\section{Organisatorische Randbedingungen}
Der Punktezähler soll möglichst günstig sein. Deshalb ist es notwendig sich auf die bestehende Hardware zu beschränken. Um das alte IPad zu unterstützen muss darauf geachtet werden, dass es sich beim Browser um eine veraltete Version des Safaribrowsers handelt.
\section{Konventionen}
Die Serversoftware soll möglichst ressourcensparend arbeiten, sodass sie auf der begrenzten Hardware des Raspberry Pi 3 läuft. Optimalerweise genügt ein Raspberry Pi Zero.